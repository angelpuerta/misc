\documentclass{resume}

\usepackage[spanish]{babel}
\usepackage{datetime}
\setmainfont{Alegreya}
\setsansfont{Rubik}

\title{Currículum Vitae}
\author{Ángel Puerta Díaz}

\name{Ángel Puerta Díaz}
\phone{*******}
\email{angelpuertadiaz@gmail.com}
\country{España}
\city{Madrid}
\git{angelpuerta}

\begin{document}

%----------------------------------------------------------------------------------------
%	EDUCATION SECTION
%----------------------------------------------------------------------------------------
\begin{rSection}{Educación}
\bf {Universidad Internacional de Valencia}
\hfill  {\footnotesize \sffamily 2020 - Presente}
\\Master en Ingeniería biomédica

\bf {Escuela de Ingeniería Informática, Oviedo}
\hfill  {\footnotesize \sffamily 2015 - 2019}
\\Grado en Ingeniería Informática del Software


\end{rSection}


%----------------------------------------------------------------------------------------
%	WORK EXPERIENCE SECTION
%----------------------------------------------------------------------------------------
\begin{rSection}{Carrera profesional}

\begin{rSubsection}{Neo9}{\bf \sffamily Abri 2021 - Presente}{Ingeniero de software}{}
\item Modelar apliaciones en el universo de los seguros bajo metodologías ágiles. 
\item Desarrollo del producto a partir de tecnologías como Angular, Spring, MongoDb y Docker.
\end{rSubsection}

\begin{rSubsection}{Sopra Steria}{\bf \sffamily Octubre 2019 - Marzo 2021 }{Solution Building}{}
\item Trabajar con equipo y empresas externas en un entorno francófono
\item Analizar sistemas legados con gran número de dependencias y actores implicados en el producto
\end{rSubsection}

\begin{rSubsection}{Grupo Iris}{Marzo 2019 - Agosto 2019}{Desarrollador Web}{}
\item Desarrollo de aplicaciones en el sector aéreo, trabajando con proveedores como Saber o Amadeus.
\end{rSubsection}

\begin{rSubsection}{Clarcat}{Enero 2019 - Febrero 2019}{Prácticas como Data Scientist}{}
\item Refinar los datos de entrada y crear paneles para el seguimiento de los KPIs
\item Definir y modelar ayudas visuales para apoyarnos en la toma de decisiones
\end{rSubsection}

\end{rSection}


%--------------------------------------------------------------------------------
%    Projects And Seminars
%-----------------------------------------------------------------------------------------------
\begin{rSection}{Proyectos}


\begin{rSubsection}{Caduceo}{}{}{}
\item Ayudar en la toma de decisión de nuevas líneas de investigación a partir de la información del transcriptoma
\item Facilitar la inclusión de nuevos modelos y algoritmos en el framework para acortar los tiempos
\item Unir el transcriptoma con variables más accesibles como posibles rutas metabólicas
\end{rSubsection}

\begin{rSubsection}{QUI?}{}{}{}
\item Estandarizar por completo el ciclo de vida de los eventos desde su creación hasta el estudio de los datos obtenidos.
\item Trabajar con un equipo multidisciplinar tanto a nivel de concepción como desarrollo.
\end{rSubsection}


\end{rSection}


%--------------------------------------------------------------------------------
%    Languages
%-----------------------------------------------------------------------------------------------

\begin{rSection}{Idiomas}{}
  Español (Nativo),
  Inglés (B2), 
  Francés (B2),
  Mandarín (A2)
\end{rSection}

\end{document}
