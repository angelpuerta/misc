\documentclass{resume}

\usepackage[spanish]{babel}
\usepackage{datetime}
\setmainfont{Alegreya}
\setsansfont{Rubik}

\title{Currículum Vitae}
\author{Ángel Puerta Díaz}

\name{Ángel Puerta Díaz}
\email{angelpuertadiaz@gmail.com}
\phone{635847718}
\city{Madrid}
\country{España}
\git{angelpuerta}

\begin{document}

%----------------------------------------------------------------------------------------
%	EDUCATION SECTION
%----------------------------------------------------------------------------------------
\begin{rSection}{Educación}
\normalsize 
{
\bf { \sffamily Universidad Internacional de Valencia}
\hfill  {\footnotesize \sffamily 2020 - 2021}
}
\rmfamily
\\Master en Ingeniería biomédica

{
\bf {\sffamily Escuela de Ingeniería Informática, Oviedo}
\hfill  {\footnotesize \sffamily 2015 - 2019}
}
\rmfamily
\\Grado en Ingeniería Informática del Software


\end{rSection}


%----------------------------------------------------------------------------------------
%	WORK EXPERIENCE SECTION
%----------------------------------------------------------------------------------------
\begin{rSection}{Carrera profesional}

\begin{rSubsection}{Neo9}{\bf \sffamily Abril 2021 - Presente}{Ingeniero de software}{}
\rmfamily
\item Durante mi tiempo en Neo9, he tenido la oportunidad de participar en diversos proyectos desde su fase inicial hasta la fase de mantenimiento.\\
He incrementado mi experiencia en los marcos ágiles mediante distintos clientes y dominios, en estos años hemos conseguido mejorar la relación entre el cliente y la empresa.\\
Asimismo, he tenido la oportunidad de desempeñar diferentes roles, ofreciendo apoyo tanto en el backend como en el frontend según las necesidades del equipo. Además, en los últimos meses he podido contribuir a la infrastructura, desarrollándome en el ámbito DevOps.\\
\\
Uno de los logros mas destacados y una de las experiencias que más me ha aportado ha sido la formación, orientación y creación de equipos en tres proyectos.\\
Algunas de las tecnologías con las que he trabajado han sido Mongo, SpringBoot, WebFlux y Angular. Para la parte de los microservicios hemos empleado tecnologías basadas en Kubernetes en AWS y GCP.\\
Mi último proyecto está relacionado con la atomización de un servicio monolítico en microservicios y la creación de un portal anexo para el cliente.
\end{rSubsection}

\begin{rSubsection}{Sopra Steria}{\bf \sffamily Octubre 2019 - Marzo 2021 }{Desarrollador fullstack}{}
\rmfamily
\item En Sopra he podido inciarme en el mundo de los proyectos más allá de España y más allá de la propia empresa. En este periodo he mejorado mis habilidades de comunicación efectiva y hemos podido adaptarnos a las necesidades del cliente.\\
Además, he tenido la responsabilidad de analizar sistemas legados con un amplio número de dependencias y actores involucrados en el producto repartidos en diferentes ubicaciones.\\
En esta etapa he trabajado con JSF, Oracle SQL, Oracle Policy Manager, Mockito.
\end{rSubsection}

\begin{rSubsection}{Grupo Iris}{\bf \sffamily Marzo 2019 - Agosto 2019 }{Desarrollador fullstack}{}
  \rmfamily
  \item Desarrollo de aplicaciones en el sector aéreo, trabajando con proveedores como Saber o Amadeus.
  \end{rSubsection}

  
\begin{rSubsection}{Clarcat}{\bf \sffamily Diciembre 2018 - Febrero 2019 }{Prácticas Data Scientist}{}
    \rmfamily
    \item Refinar los datos de entrada y crear paneles para el seguimiento de los KPIs.\\
     Definir y modelar ayudas visuales para apoyarnos en la toma de decisiones
    \end{rSubsection}
    


\end{rSection}


%--------------------------------------------------------------------------------
%    Languages
%-----------------------------------------------------------------------------------------------

\begin{rSection}{Idiomas}{}
  \normalsize
  \rmfamily
  Español (Nativo),
  Inglés (B2), 
  Francés (B2),
  Mandarín (A2)
\end{rSection}

\end{document}
